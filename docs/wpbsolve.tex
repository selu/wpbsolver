% !TEX TS-program = pdflatex
% !TEX encoding = UTF-8 Unicode

% This is a simple template for a LaTeX document using the "article" class.
% See "book", "report", "letter" for other types of document.

\documentclass[11pt]{article} % use larger type; default would be 10pt

\usepackage[utf8]{inputenc} % set input encoding (not needed with XeLaTeX)

%%% Examples of Article customizations
% These packages are optional, depending whether you want the features they provide.
% See the LaTeX Companion or other references for full information.

%%% PAGE DIMENSIONS
\usepackage{geometry} % to change the page dimensions
\geometry{a4paper} % or letterpaper (US) or a5paper or....
% \geometry{margin=2in} % for example, change the margins to 2 inches all round
% \geometry{landscape} % set up the page for landscape
%   read geometry.pdf for detailed page layout information

\usepackage{graphicx} % support the \includegraphics command and options

% \usepackage[parfill]{parskip} % Activate to begin paragraphs with an empty line rather than an indent

%%% PACKAGES
\usepackage{booktabs} % for much better looking tables
\usepackage{array} % for better arrays (eg matrices) in maths
\usepackage{paralist} % very flexible & customisable lists (eg. enumerate/itemize, etc.)
\usepackage{verbatim} % adds environment for commenting out blocks of text & for better verbatim
\usepackage{subfig} % make it possible to include more than one captioned figure/table in a single float
% These packages are all incorporated in the memoir class to one degree or another...

%%% HEADERS & FOOTERS
\usepackage{fancyhdr} % This should be set AFTER setting up the page geometry
\pagestyle{fancy} % options: empty , plain , fancy
\renewcommand{\headrulewidth}{0pt} % customise the layout...
\lhead{}\chead{}\rhead{}
\lfoot{}\cfoot{\thepage}\rfoot{}

%%% SECTION TITLE APPEARANCE
\usepackage{sectsty}
\allsectionsfont{\sffamily\mdseries\upshape} % (See the fntguide.pdf for font help)
% (This matches ConTeXt defaults)

%%% ToC (table of contents) APPEARANCE
\usepackage[nottoc,notlof,notlot]{tocbibind} % Put the bibliography in the ToC
\usepackage[titles,subfigure]{tocloft} % Alter the style of the Table of Contents
\renewcommand{\cftsecfont}{\rmfamily\mdseries\upshape}
\renewcommand{\cftsecpagefont}{\rmfamily\mdseries\upshape} % No bold!

\usepackage{hyperref}
\usepackage{mathtools}
\usepackage{numprint}
\usepackage[usenames,x11names,table]{xcolor}
\usepackage{minted}
\usepackage{tikz}
\usetikzlibrary{trees}
\usetikzlibrary{calc}
\usepackage{draftwatermark}
\SetWatermarkScale{5}
\usepackage[markings,customcolors]{hf-tikz}
\usepackage{fontawesome}
\usepackage{pgfplots}

% Code from Christian Feuersänger
% http://tex.stackexchange.com/questions/54794/using-a-pgfplots-style-legend-in-a-plain-old-tikzpicture#54834

% argument #1: any options
\newenvironment{customlegend}[1][]{%
    \begingroup
    % inits/clears the lists (which might be populated from previous
    % axes):
    \csname pgfplots@init@cleared@structures\endcsname
    \pgfplotsset{#1}%
}{%
    % draws the legend:
    \csname pgfplots@createlegend\endcsname
    \endgroup
}%

% makes \addlegendimage available (typically only available within an
% axis environment):
\def\addlegendimage{\csname pgfplots@addlegendimage\endcsname}

%%% END Article customizations

%%% The "real" document content comes below...

\title{Solve Weighing Pool Ball Puzzle}
\author{Szabolcs Seláf \\ \texttt{selu@selu.org}}
%\date{} % Activate to display a given date or no date (if empty),
         % otherwise the current date is printed 

\begin{document}
\maketitle

\section{Weighing Pool Ball Puzzle}

The problem is described at \url{https://www.mathsisfun.com/pool_balls.html} with a solution. But the given solution is undetermined. Subsequent measures are depends on the result of the previous one. Ferenc Rákóczi mentioned, he already had a determined solution for this problem, but he forgot it.

\subsection{The problem}

You have 12 balls identical in size and appearance but 1 is an odd weight (could be either light or heavy).

You have a set of balance scales which will give 3 possible readings:

\begin{itemize}
\item Left $=$ Right
\item Left $>$ Right
\item Left $<$ Right (in example Left and Right have equal weight, Left is Heavier, or Left is Lighter).
\end{itemize}

You have \emph{only 3 chances} to weigh the balls in any combination using the scales.
Find which ball is the odd one and if it's heavier or lighter than the rest.

\section{Find one solution}

\subsection{Generate possible measures} \label{one_solution}

At first I calculated the number of possible measures. We should put equal number of balls onto both arms to get a valid result, so pick even number of balls for measure, then pick half of the balls for Left arm and put remaining balls into Right arm. We can halve the numbers, because putting same set of balls to Right arm is same as to Left arm:

\[ \sum_{i=1}^{6} \frac{\binom{12}{2i} \binom{2i}{i}}{2} = \numprint{36894} \]

All three measures can be one of the \numprint{36894} possible ones, so the number of all possible solutions could be:

\[ \numprint{36894}^3 = \numprint{50218904004984} \approx \numprint{5e13} \]

Using information above we can generate possible measures one-by-one:

\begin{minted}{ruby}
ids = (1..12).to_a                          # array [1, 2, .., 12] for balls
(1..6).each do |level|                      # how much balls to put on an arm
  ids.combination(level*2).each do |subset| # pick a combination for both arms
    reverse = []
    subset.combination(level).each do |left| # pick half of them for left arm
      next if reverse.include?(left)         # go to next if reversed already
      right = subset - left                  # checked
      reverse << right
\end{minted}

\subsection{Backtracking}

At start any of the 12 balls can be heavier or lighter than others, which means we have 24 possible results. Each measure has 3 possible outcomes, 3 measures could solve $3^3 = 27$ different cases. After the first measure the remaining two could solve only $3^2 = 9$ cases and of course, the last one could solve maximum 3 cases.

The program uses this knowledge to give up, once there is a possible outcome of a measure with more different cases, than that can be eliminated by the remaining steps.

\begin{minted}{ruby}
if state_set.map(&:case_number).max > (Scale.number_of_outcomes ** max_measures)
  return
end
\end{minted}

\subsection{Result} \label{one_result}

However this algorithm is almost a brute force, found the first solution within some seconds.

\begin{table}[h]
\begin{center}
\begin{tabular}{c|cccc|cccc}
  \toprule
  Number & \multicolumn{4}{c|}{Left arm} & \multicolumn{4}{c}{Right arm} \\
  \midrule
       1 & 1 & 2 & 3 & 4 & 5 & 6 & 7 & 8 \\
       2 & 1 & 2 & 3 & 5 & 4 & 9 & 10 & 11 \\
       3 & 1 & 4 & 6 & 9 & 2 & 7 & 10 & 12 \\
  \bottomrule
\end{tabular}
\end{center}
\caption{Measures}
\end{table}

\pgfmathsetmacro{\angleA}{360/3}
\pgfmathsetmacro{\angleB}{360/9}
\pgfmathsetmacro{\angleC}{360/27}
\newlength{\leveldistance}\newlength{\longBdistance}\newlength{\longCdistance}
\newlength{\discthick}
\pgfmathsetlength{\leveldistance}{1.6cm}
\pgfmathsetlength{\discthick}{0.8cm}
\pgfmathsetlength{\longBdistance}{sqrt((2*\leveldistance*sin(\angleB))^2
                     +(2*\leveldistance*cos(\angleB)-\leveldistance)^2)}
\pgfmathsetlength{\longCdistance}{sqrt((3*\leveldistance*sin(\angleC))^2
                     +(3*\leveldistance*cos(\angleC)-2*\leveldistance)^2)}
\pgfmathsetmacro{\angleBmod}{asin(2*\leveldistance*sin(\angleB)/\longBdistance)}
\pgfmathsetmacro{\angleCmod}{asin(3*\leveldistance*sin(\angleC)/\longCdistance)}

\begin{figure}[h]
\begin{tikzpicture}[
  ball/.style       = {shape=circle, top color=white, minimum size=0.8cm},
  small/.style       = {minimum size=0.5cm},
  choose1/.style     = {ball, top color=brown!30!white, bottom color=brown!70!black, text=yellow!30!white},
  choose2/.style     = {ball, top color=brown!40!white, bottom color=brown!60!black, text=yellow!30!white},
  choose3/.style     = {ball, top color=brown!50!white, bottom color=brown!50!black, text=yellow!30!white},
  heavy/.style      = {ball, bottom color=red!50!black, top color=red!30!white, text=white},
  light/.style      = {ball, top color=blue!10, bottom color=blue!30},
  nothing/.style    = {ball, bottom color=magenta!30!black, top color=magenta!30!white},
  heavier/.style    = {draw=red, ->},
  lighter/.style    = {draw=blue, ->},
  equal/.style      = {draw=gray, ->},
  grow cyclic, thick,
  level/.style      = { level distance=\leveldistance },
  long2/.style      = { level distance=\longBdistance },
  long3/.style      = { level distance=\longCdistance },
  level 1/.style    = { sibling angle=\angleA },
  level 2/.style    = { sibling angle=\angleBmod },
  level 3/.style    = { sibling angle=\angleCmod },
  lightrot/.style   = { rotate=\angleBmod-\angleB },
  heavyrot/.style   = { rotate=\angleB-\angleBmod }
]
\fill[teal!40] circle(\discthick);
\foreach \i in {1,...,3}
{
  \pgfmathsetmacro{\colorl}{40-\i*10}
  \fill[even odd rule, teal!\colorl]
      circle(\leveldistance*\i+\discthick)
      circle(\leveldistance*\i-\discthick);
}
\node [choose1] (l1) {\faBalanceScale}
  child { node [choose2] (l13) {\faBalanceScale}
    child [long2] { node [choose3] (l133) {\faBalanceScale}
      child [lightrot,long3] { node [light] {1} edge from parent [lighter] }
      child [lightrot] { node [light] {3} edge from parent [equal] }
      child [lightrot,long3] { node [light] {2} edge from parent [heavier] }
      edge from parent [lighter]
    }
    child { node [choose3] (l132) {\faBalanceScale} 
      child [long3] { node [heavy] {7} edge from parent [lighter] }
      child { node [heavy] {8} edge from parent [equal] }
      child [long3] { node [heavy] {6} edge from parent [heavier] }
      edge from parent [equal]
    }
    child [long2] { node [choose3] (l131) {\faBalanceScale} 
      child [heavyrot,long3] { node [light] {4} edge from parent [lighter] }
      child [heavyrot] { node [heavy] {5} edge from parent [equal] }
      child [heavyrot,long3] { node [nothing] {} edge from parent [heavier] }
      edge from parent [heavier]
    }
    edge from parent [lighter]
  }
  child { node [choose2] (l12) {\faBalanceScale}
    child [long2] { node [choose3] (l123) {\faBalanceScale} 
      child [lightrot,long3] { node [heavy] {10} edge from parent [lighter] }
      child [lightrot] { node [heavy] {11} edge from parent [equal] }
      child [lightrot,long3] { node [heavy] {9} edge from parent [heavier] }
      edge from parent [lighter]
    }
    child { node [choose3] (l122) {\faBalanceScale} 
      child [long3] { node [heavy] {12} edge from parent [lighter] }
      child { node [nothing] {} edge from parent [equal] }
      child [long3] { node [light] {12} edge from parent [heavier] }
      edge from parent [equal]
    }
    child [long2] { node [choose3] (l121) {\faBalanceScale} 
      child [heavyrot,long3] { node [light] {9} edge from parent [lighter] }
      child [heavyrot] { node [light] {11} edge from parent [equal] }
      child [heavyrot,long3] { node [light] {10} edge from parent [heavier] }
      edge from parent [heavier]
    }
    edge from parent [equal]
  }
  child { node [choose2] (l11) {\faBalanceScale}
    child [long2] { node [choose3] (l113) {\faBalanceScale}
      child [lightrot,long3] { node [nothing] {} edge from parent [lighter] }
      child [lightrot] { node [light] {5} edge from parent [equal] }
      child [lightrot,long3] { node [heavy] {4} edge from parent [heavier] }
      edge from parent [lighter]
    }
    child { node [choose3] (l112) {\faBalanceScale}
      child [long3] { node [light] {6} edge from parent [lighter] }
      child { node [light] {8} edge from parent [equal] }
      child [long3] { node [light] {7} edge from parent [heavier] }
      edge from parent [equal]
    }
    child [long2] { node [choose3] (l111) {\faBalanceScale}
      child [heavyrot,long3] { node [heavy] {2} edge from parent [lighter] }
      child [heavyrot] { node [heavy] {3} edge from parent [equal] }
      child [heavyrot,long3] { node [heavy] {1} edge from parent [heavier] }
      edge from parent [heavier]
    }
    edge from parent [heavier]
  }
  ;
\pgfplotsset{
    legend ball with text/.style n args={2}{
        legend image code/.code={%
            \node[#2,small] {#1};
        }
    },
	}
\begin{customlegend}[legend cell align=left, %<= to align cells
legend entries={ % <= in the following there are the entries
left heavier,
left lighter,
equal,
heavier ball,
lighter ball,
impossible
},
legend style={at={(7.5,-2.5)},font=\footnotesize}] % <= to define position and font legend
% the following are the "images" and numbers in the legend
    \addlegendimage{heavier}
    \addlegendimage{lighter}
    \addlegendimage{equal}
    \addlegendimage{legend ball with text={1}{heavy}}
    \addlegendimage{legend ball with text={2}{light}}
    \addlegendimage{legend ball with text={}{nothing}}
\end{customlegend}
\end{tikzpicture}
\caption{Decision Tree}
\end{figure}

\begin{table}[h]
\begin{center}
\begin{tabular}{ccc}

\begin{tabular}{cc}
\toprule
Measures & Result \\
\midrule
$>>>$ & $\uparrow 1$ \\
$<<<$ & $\downarrow 1$ \\

$>><$ & $\uparrow 2$ \\
$<<>$ & $\downarrow 2$ \\

$>>=$ & $\uparrow 3$ \\
$<<=$ & $\downarrow 3$ \\

$><>$ & $\uparrow 4$ \\
$<><$ & $\downarrow 4$ \\

$===$ & $\emptyset$ \\
\bottomrule
\end{tabular} 
&
\begin{tabular}{cc}
\toprule
Measures & Result \\
\midrule
$<>=$ & $\uparrow 5$ \\
$><=$ & $\downarrow 5$ \\

$<=>$ & $\uparrow 6$ \\
$>=<$ & $\downarrow 6$ \\

$<=<$ & $\uparrow 7$ \\
$>=>$ & $\downarrow 7$ \\

$<==$ & $\uparrow 8$ \\
$>==$ & $\downarrow 8$ \\

$><<$ & $\emptyset$ \\
\bottomrule
\end{tabular} 
&
\begin{tabular}{cc}
\toprule
Measures & Result \\
\midrule
$=<>$ & $\uparrow 9$ \\
$=><$ & $\downarrow 9$ \\

$=<<$ & $\uparrow 10$ \\
$=>>$ & $\downarrow 10$ \\

$=<=$ & $\uparrow 11$ \\
$=>=$ & $\downarrow 11$ \\

$==<$ & $\uparrow 12$ \\
$==>$ & $\downarrow 12$ \\

$<>>$ & $\emptyset$ \\
\bottomrule
\end{tabular}

\end{tabular}
\end{center}
\caption{Results}
\end{table}

\section{Find all solutions}

Let's continue the calculation, what we started in subsection \ref{one_solution}.

\subsection{How many different measures are possible?} \label{how_many}

We have $\numprint{36894}^3$ case, what we have to check to find all solutions, but it is too many, so we try to reduce this number with some ideas. At first, the $3$ measures should be different, why we should do same measure twice? Also because any of the subsequent measure(s) should not depend on previous one(s), the order of them is indifferent, so we can calculate all cases as

\[ \binom{\numprint{36894}}{3} = \numprint{8369136762844} \approx \numprint{8.37e12} \]

This is a bit better, but also too many for check all of them. Examine, how many measures of $\numprint{36894}$ may lead to the right results. It seems from the solution what we found in subsection \ref{one_result} it works, when we put $\frac{1}{3}$ of the balls onto each arm.

Let's see what happens, when we put less than $4$ balls. If we put $6$ balls - $3$-$3$ onto each arm - and they would be equal, we won't know anything about the remaining $6$ balls, what means $12$ possible outcomes for the remaining two measures. With less balls it could be worse, so we should measure at least $4$ balls. 

Now, try to measure more than $8$ balls, in example $10$. In this case, if they are not equal, half of them can be lighter and another half of them can be heavier, what is $10$ different possible result for the remaining two measures. We can say, only those measures can work when we put $\frac{1}{3}$ of the balls onto each arm, in our case they are

\[ \frac{\binom{12}{8}\binom{8}{4}}{2} = \numprint{17325} \]

so all cases, what we need to check now

\[ \binom{\numprint{17325}}{3} = \numprint{866549295150} \approx \numprint{8.67e11} \]

It is much better than when we started, but even too many, so let's try another approach.

\subsection{Series of measures by balls}

Every balls can be in three position in each measures,
\begin{itemize}
\item ball on the left arm - with numbers $1$,
\item ball on the right arm - $-1$,
\item ball is not measured - $0$
\end{itemize}
So there are $3^3 = 27$ different series of measures, what is the repeated permutations of three positions above for $3$ measures, how a ball can be placed in the three measures. One case, when ball is not measured at all $\left[\begin{smallmatrix}0\\0\\0\end{smallmatrix}\right]$ should be left out, of course, so remained $26$ of them. We know one more thing, we should measure every balls differently, so we should simply choose $12$ different element from these $26$, what is

\[ \binom{26}{12} = \numprint{9657700} \]

This is much better, than any other approach and we could easily check all of them with a computer, but let's see whether we can reduce this number better.

There are measure series, what are mirrors of each other, which means both balls are measured or not, and if they are measured, they are always on the opposite side of the balance, in example $\left[\begin{smallmatrix}\phantom{{}-{}}1\\-1\\\phantom{{}-{}}0\end{smallmatrix}\right]$ and $\left[\begin{smallmatrix}-1\\\phantom{{}-{}}1\\\phantom{{}-{}}0\end{smallmatrix}\right]$ are mirrors of each other. It is simply to find the mirrors, just multiply each elements by $-1$. In fact every element have mirrors, so finally we have only $\frac{26}{2} = 13$ different series of measures, see figure~\ref{fig:table_series}.

\begin{figure}[h]
\caption{Series of measures}
\label{fig:table_series}
\tikzset{
   pwrong/.style={set border color=red, set fill color=red!10!white},
   pmain/.style={set border color=green!50!black, set fill color=white},
   pmirror/.style={set border color=blue, set fill color=blue!10!white}
}
\[
\begin{array}{rrrrrrrrr}
\tikzmarkin[pmain]{p1} \phantom{{}-{}}1 &
\tikzmarkin[pmain]{p2} \phantom{{}-{}}1 &
\tikzmarkin[pmain]{p3} \phantom{{}-{}}1 &
\tikzmarkin[pmain]{p4} \phantom{{}-{}}1 &
\tikzmarkin[pmain]{p5} \phantom{{}-{}}1 &
\tikzmarkin[pmain]{p6} \phantom{{}-{}}1 &
\tikzmarkin[pmain]{p7} \phantom{{}-{}}1 &
\tikzmarkin[pmain]{p8} \phantom{{}-{}}1 &
\tikzmarkin[pmain]{p9} \phantom{{}-{}}1 \\
 1 &  1 &  1 & -1 & -1 & -1 &  0 &  0 &  0 \\
 1 \tikzmarkend{p1} &
-1 \tikzmarkend{p2} &
 0 \tikzmarkend{p3} &
 1 \tikzmarkend{p4} &
-1 \tikzmarkend{p5} &
 0 \tikzmarkend{p6} &
 1 \tikzmarkend{p7} &
-1 \tikzmarkend{p8} &
 0 \tikzmarkend{p9} \\
 \\ \quad \\
\tikzmarkin[pmirror]{m5}-1 &
\tikzmarkin[pmirror]{m4}-1 &
\tikzmarkin[pmirror]{m6}-1 &
\tikzmarkin[pmirror]{m2}-1 &
\tikzmarkin[pmirror]{m1}-1 &
\tikzmarkin[pmirror]{m3}-1 &
\tikzmarkin[pmirror]{m8}-1 &
\tikzmarkin[pmirror]{m7}-1 &
\tikzmarkin[pmirror]{m9}-1 \\
 1 &  1 &  1 & -1 & -1 & -1 &  0 &  0 &  0 \\
 1 \tikzmarkend{m5} &
-1 \tikzmarkend{m4} &
 0 \tikzmarkend{m6} &
 1 \tikzmarkend{m2} &
-1 \tikzmarkend{m1} &
 0 \tikzmarkend{m3} &
 1 \tikzmarkend{m8} &
-1 \tikzmarkend{m7} &
 0 \tikzmarkend{m9} \\
 \\ \quad \\
\tikzmarkin[pmain]{p10} \phantom{{}-{}}0 &
\tikzmarkin[pmain]{p11} \phantom{{}-{}}0 &
\tikzmarkin[pmain]{p12} \phantom{{}-{}}0 &
\tikzmarkin[pmirror]{m11} \phantom{{}-{}}0 &
\tikzmarkin[pmirror]{m10} \phantom{{}-{}}0 &
\tikzmarkin[pmirror]{m12} \phantom{{}-{}}0 &
\tikzmarkin[pmain]{p13} \phantom{{}-{}}0 &
\tikzmarkin[pmirror]{m13} \phantom{{}-{}}0 &
\tikzmarkin[pwrong]{wrong} \phantom{{}-{}}0 \\
 1 &  1 &  1 & -1 & -1 & -1 &  0 &  0 &  0 \\
 1 \tikzmarkend{p10} &
-1 \tikzmarkend{p11} &
 0 \tikzmarkend{p12} &
 1 \tikzmarkend{m11} &
-1 \tikzmarkend{m10} &
 0 \tikzmarkend{m12} &
 1 \tikzmarkend{p13} &
-1 \tikzmarkend{m13} &
 0 \tikzmarkend{wrong} 
\end{array}
\]
\begin{tikzpicture}[overlay,remember picture]
  \draw[<->,cyan,bend left] ($(p10)+(0.5,0.03)$) edge ($(m10)+(0.5,0.03)$);
  \draw[<->,cyan,bend left] ($(p11)+(0.5,0.03)$) edge ($(m11)+(0.5,0.03)$);
  \draw[<->,cyan,bend left] ($(p12)+(0.5,0.03)$) edge ($(m12)+(0.5,0.03)$);
  \draw[<->,cyan,bend left] ($(p13)+(0.5,0.03)$) edge ($(m13)+(0.5,0.03)$);
  \draw[<->,cyan] ($(p1)+(0.5,-1.53)$) edge ($(m1)+(0.5,0.03)$);
  \draw[<->,cyan] ($(p2)+(0.5,-1.53)$) edge ($(m2)+(0.5,0.03)$);
  \draw[<->,cyan] ($(p3)+(0.5,-1.53)$) edge ($(m3)+(0.5,0.03)$);
  \draw[<->,cyan] ($(p4)+(0.5,-1.53)$) edge ($(m4)+(0.5,0.03)$);
  \draw[<->,cyan] ($(p5)+(0.5,-1.53)$) edge ($(m5)+(0.5,0.03)$);
  \draw[<->,cyan] ($(p6)+(0.5,-1.53)$) edge ($(m6)+(0.5,0.03)$);
  \draw[<->,cyan] ($(p7)+(0.5,-1.53)$) edge ($(m7)+(0.5,0.03)$);
  \draw[<->,cyan] ($(p8)+(0.5,-1.53)$) edge ($(m8)+(0.5,0.03)$);
  \draw[<->,cyan] ($(p9)+(0.5,-1.53)$) edge ($(m9)+(0.5,0.03)$);
\end{tikzpicture}
\end{figure}

In this case we can choose $12$ series of those $13$, then try any combination of their mirrored and non-mirrored version

\[ \binom{13}{12} 2^{12} = \numprint{53248} \]

cases in total, what is a wonderful number, but we know one more thing from subsection~\ref{how_many}. We should measure only $8$ balls in each step, the combination of chosen series of measures should have exactly $4$ zeros in each row. Because the base $13$ series have no any extra zeros, when we choose the $12$ series, the one to leave can only one of the $4$ series without any zeros. Which means our number is

\[ \binom{4}{3} 2^{12} = \numprint{16384} \]

If we check all these cases, we find there are only $308$ good solutions. However we can refine our calculation more, because the sum for each measure should be $0$, which means same number of balls should be on both arms of balance. So we should choose $4$ balls to mirror from the first $8$, also the last two series $\left[\begin{smallmatrix}0\\0\\1\end{smallmatrix}\right]$ and $\left[\begin{smallmatrix}\phantom{{}-{}}0\\\phantom{{}-{}}0\\-1\end{smallmatrix}\right]$ are determined by others. The other two can be anything based on the earlier values, so for then we can use $2^2$ and here is the result:

\[ \binom{4}{3} \binom{8}{4} 2^2 = \numprint{1120} \]

\subsection{How many solutions are non-equivalent}

If we check the found $308$ solutions, we can see, there are some of them, what are mirrors of another ones\ldots

\[
\begin{bmatrix*}[r]
1 \\ & 1 \\ && -1
\end{bmatrix*}
\begin{bmatrix*}[r]
1 & 1 & 1 \\ 1 & 1 & -1 \\ 1 & -1 & 1
\end{bmatrix*} =
\begin{bmatrix*}[r]
1 & 1 & 1 \\ 1 & 1 & -1 \\ -1 & 1 & -1
\end{bmatrix*}
\]

\begin{tabular}{!{\color{Orange2}\vrule width 2pt}>{\color{RoyalBlue3}}r!{\color{Orange2}\vrule width 2pt}>{\color{RoyalBlue3}}r!{\color{Orange2}\vrule width 2pt}>{\color{RoyalBlue3}}r!{\color{Orange2}\vrule width 2pt}}
\arrayrulecolor{Orange2}\specialrule{2pt}{0pt}{0pt}
1 & 1 & 1 \\\arrayrulecolor{Orange2}\specialrule{2pt}{0pt}{0pt}
1 & 1 & -1 \\\arrayrulecolor{Orange2}\specialrule{2pt}{0pt}{0pt}
1 & -1 & 1 \\\arrayrulecolor{Orange2}\specialrule{2pt}{0pt}{0pt}
\end{tabular}
\end{document}
